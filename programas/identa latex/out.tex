\documentclass[a4paper, 12pt]{article}

\usepackage[portuges]{babel}
\usepackage[utf8]{inputenc}
\usepackage{indentfirst}
\usepackage{graphicx}
\usepackage{float}% ante frescura de imagens
\usepackage{multicol,lipsum}
\usepackage{mathptmx}
\usepackage{ragged2e}% testo justificado
\usepackage{setspace} % espaçamento entre linhas
\usepackage{amsmath} % números complexos
\usepackage{amssymb} % mesure angle
\usepackage{siunitx} %simbolo micro

\usepackage{geometry}
\geometry{ a4paper, total={170mm,257mm}, left=30mm, right = 25mm, top=30mm, bottom = 20mm }

% padrao 1.5 de espacamento entre linhas
\setstretch{1.5}
\begin{document}
	
	%\maketitle
	
	\begin{titlepage}
		\begin{center}
			\Huge{Universidade Federal de Uberlândia}\\
			\textbf{\LARGE{}}\\
			%\title{{\large{Título}}}
			\vspace{3,5cm}
		\end{center}
		
		\begin{flushleft}
			\begin{tabbing}
				Aluno: Henrique Santos de Lima - 11811ETE016\\
				Professor: Wellington Maycon Santos Bernardes\\
			\end{tabbing}
		\end{flushleft}
		\vspace{1cm}
		
		\begin{center}
			\vspace{\fill}
			Novembro\\
			2019
		\end{center}
	\end{titlepage}
	%%%%%%%%%%%%%%%%%%%%%%%%%%%%%%%%%%%%%%%%%%%%%%%%%%%%%%%%%%%
	
	% % % % % % % % %FOLHA DE ROSTO % % % % % % % % % %
	
	\begin{titlepage}
		\begin{center}
			
			%\begin{figure}[!ht]
				%\centering
				%\includegraphics[width=2cm]{c:/ufba.jpg}
			%\end{figure}
			
			\Huge{Universidade Federal de Uberlândia}\\
			\vspace{15pt}
			
			\vspace{85pt}
			
			\textbf{\LARGE{Relatório de Experimental de Circuitos Elétricos 2}}
			\title{\large{Título}}
			%	\large{Modelo\\
			%   		Validação do modelo clássico}
			
		\end{center}
		\vspace{1,5cm}
		
		\begin{flushright}
			\begin{list}{}{
				\setlength{\leftmargin}{4.5cm}
				\setlength{\rightmargin}{0cm}
				\setlength{\labelwidth}{0pt}
				\setlength{\labelsep}{\leftmargin}}
				
				\item
				Circuitos Trifásicos desequilibrados
				\begin{list}{}{
					\setlength{\leftmargin}{0cm}
					\setlength{\rightmargin}{0cm}
					\setlength{\labelwidth}{0pt}
					\setlength{\labelsep}{\leftmargin}}
					\item Aluno:  Henrique Santos de Lima - 11811ETE016\
					\item Professor : Wellington Maycon Santos Bernardes\
				\end{list}
			\end{list}
		\end{flushright}
		\vspace{1cm}
		\begin{center}
			\vspace{\fill}
			Novembro\\
			2019
		\end{center}
	\end{titlepage}
	
	\tableofcontents
	
	\thispagestyle{empty}
	
	\newpage
	%\pagenumbering{times new romam}
	% % % % % % % % % % % % % % % % % % % % % % % % % % %
	\section{Objetivos}
	
	\justifying
	Medir as potência ativa e reativa usando o método dos dois wattímetros, averiguar se este método funciona comparando com os valores obtidos analiticamente.
	\section{Introdução}
	
	\justifying
	O Wattímetro analógico mostra com sua ponteira o resultado da parte real da multiplicação
	da tensão lida com o conjugado da corrente lida. \(W = \Re\{V \ast I^\ast\}\) . Para uma mesma fase essa leitura possui significado físico de potencia ativa, porém dependendo da ligação a medida apresentada pode não ter um significado físico.
	
	O método dos dois wattímetros consiste em usar dois wattímetros para realizar a medida da potencia trifásica. Para realizar essa medida deve-se :
	\begin{itemize}
		\item Escolher duas fases.
		\item Medir a corrente que passa na  fase escolhida e a tensão entre a mesma e a fase não \\ escolhida.
		\item Fazer o mesmo com a segunda Fase escolhida.
	\end{itemize}
	Exemplo:\\
	\begin{figure}[H]
		\centering % para centralizarmos a figura
		\includegraphics[]{montagem_2watt_A+C.png}
		\caption{montagem usando 2 wattímetros}
	\end{figure}
	
	Para entender qual significado físico de cada medição:
	Considerando a sequencia ABC temos:
	\[ \begin{split}
		&\begin{bmatrix} V_{AB}\\ V_{BC}\\ V_{CA} \end{bmatrix} = V_L \angle \theta \begin{bmatrix} 1\\  \alpha^2\\  \alpha \end{bmatrix}\\
		&\begin{bmatrix} V_{AB}\\ V_{BC}\\ V_{CA} \end{bmatrix} = V_L \angle \theta \begin{bmatrix} 1\\  \alpha^2\\  \alpha \end{bmatrix}\\
		&\begin{bmatrix} V_{AN}\\ V_{BN}\\ V_{CN} \end{bmatrix} = \frac{V_L}{\sqrt{3}} \measuredangle{( \theta-30^\circ)} \begin{bmatrix} 1\\  \alpha^2\\  \alpha \end{bmatrix} \\
		&\begin{bmatrix} V_{AN}\\ V_{BN}\\ V_{CN} \end{bmatrix} = \frac{V_L}{\sqrt{3}} \measuredangle{( \theta-30^\circ)} \begin{bmatrix} 1\\  \alpha^2\\  \alpha \end{bmatrix} \\
		&\begin{bmatrix} I_{A}\\ I_{B}\\ I_{C} \end{bmatrix} = I_L\measuredangle{(( \theta-30^\circ) - \theta_z)} \begin{bmatrix} 1\\  \alpha^2\\  \alpha \end{bmatrix}
		&\begin{bmatrix} I_{A}\\ I_{B}\\ I_{C} \end{bmatrix} = I_L\measuredangle{(( \theta-30^\circ) - \theta_z)} \begin{bmatrix} 1\\  \alpha^2\\  \alpha \end{bmatrix}
	\end{split}
	\]
	
	\[\begin{split}
		W_1 &= \Re\{V_{AC} \ast I_A^\ast\}\\
		W_1 &= \Re\{-\alpha \ast V_L\measuredangle{\theta} \ast I_L\measuredangle{(-\theta+30^\circ+\theta_z)}\}\\
		W_1 &= \Re\{V_L\ast I_L\measuredangle{(\theta-\theta+30^\circ+\theta_z - 60^\circ)}\}\\
		W_1 &= \Re\{V_L\ast I_L\measuredangle{(\theta_z - 30^\circ)}\}\\
		W_1 &= V_L\ast I_L \ast \cos(\theta _z - 30^\circ)
	\end{split}\]
	\[
	\begin{split}
		W_2 &= \Re\{V_{BC} \ast I_B^\ast\}\\
		W_2 &= \Re\{(\alpha^2 \ast V_L\measuredangle{\theta})\ast (\alpha^2 \ast I_L\measuredangle{(\theta-30^\circ-\theta_z)}\})^\ast\\
		W_2 &= \Re\{(V_L\measuredangle{\{\theta-120^\circ}\})\ast (\ast I_L\measuredangle{(-\theta+30^\circ+\theta_z+120^\circ)}\})\\
		W_2 &= \Re\{V_L\ast I_L\measuredangle{(\theta-\theta+30^\circ+\theta_z-120^\circ + 120^\circ)}\}\\
		W_2 &= \Re\{V_L\ast I_L\measuredangle{(\theta_z + 30^\circ)}\}\\
		W_2 &= V_L\ast I_L \ast \cos(\theta _z + 30^\circ)
	\end{split}\]
	
	De modo que $W_1 + W_2 = \sqrt{3}*V_L*I_L*\cos(\theta_z) =P_{3\phi} $ que é a potencia trifásica da carga, e $W_1 - W_2 = V_L*I_L*\sin(\theta_z) = \frac{Q_{3\phi}}{\sqrt{3}}$
	
	Para determinar qual wattímetro corresponde ao $W_1$ e $W_2$ utiliza-se o método abaixo:
	\newpage
	\begin{itemize}
		\item Desenhar a sequencia de fase
	\end{itemize}
	\begin{figure}[H]
		\centering % para centralizarmos a figura
		\includegraphics[width=0.5\columnwidth]{definindo_w1_e_w2_p1.png}
		\caption{Desenho para sequencia ABC}
	\end{figure}
	\begin{itemize}
		\item Destacar fase que não possui Wattímetro
	\end{itemize}
	\begin{figure}[H]
		\centering % para centralizarmos a figura
		\includegraphics[width=0.5\columnwidth]{definindo_w1_e_w2_p2.png}
		\caption{Fase C destacada}
	\end{figure}
	
	\newpage
	\begin{itemize}
		\item Girar no sentido Horário, o primeiro encontrado é o $W_1$ e o segundo é o $W_2$
	\end{itemize}
	\begin{figure}[H]
		\centering % para centralizarmos a figura
		\includegraphics[width=0.5\columnwidth]{definindo_w1_e_w2_p3.png}
		\caption{$W_A$ foi encontrado primeiro, logo $W_A = W_1$ e $W_B = W_2$}
	\end{figure}
	\newpage
	\section{Preparação}
	\subsection{Materiais e ferramentas}
	\begin{itemize}
		\item Regulador de tensão(Varivolt)
		\item Resistores banana de 50\(\Omega\)
		\item Indutor de 160 mH
		\item Capacitor de 45.9 \si{\micro}F
		\item Medidor Trifásico Kron Mult-K
		\item Amperímetro analógico AC
		\item Wattímetro analógico
	\end{itemize}
	\subsection{Montagem}
	\subsubsection{Ligação em Estrela}
	\begin{figure}[H]
		\centering % para centralizarmos a figura
		\includegraphics[width = 15cm]{montagem1.png}
		\caption{circuito em estrela a ser montado}
	\end{figure}
	Para realizar a montagem deve seguir a figura 5, antes de iniciar a montagem certifique-se que o circuito esteja desligado.
	\begin{figure}[H]
		\centering % para centralizarmos a figura
		\includegraphics[width= 14cm]{my_montagem1.jpeg}
		\caption{circuito estrela montado}
	\end{figure}
	
	\subsubsection{Ligação em Delta}
	\begin{figure}[H]
		\centering % para centralizarmos a figura
		\includegraphics[width= 15cm]{montagem2.png}
		\caption{circuito em delta a ser montado}
		\label{figura:montada}
	\end{figure}
	
	Para realizar a montagem deve seguir a figura 7, antes de iniciar a montagem certifique-se que o circuito esteja desligado.
	\begin{figure}[H]
		\centering % para centralizarmos a figura
		\includegraphics[width= 14cm]{my_montagem2.jpeg}
		\caption{circuito delta montado}
	\end{figure}
	
	
	
	\newpage
	\section{Análise sobre segurança}
	\mbox{}
	\justifying
	Antes de montar o experimento é importante o uso de equipamentos de proteção, estar com calça, sapatos fechados, sem acessórios metálicos e se o cabelo for grande, este deve estar preso.
	
	A bancada deve estar desenergizada durante a montagem. Durante o experimento não ter contato com nenhum fio ou elemento energizado do circuito além do risco de choque elétrico. Certifique-se de que os equipamentos estão na escala adequada para realizar as medições.
	
	Para movimentar os indutores pegue pela parte inferior evitando riscos de que se desprenda e caia, assim evitando lesões e dano ao dispositivo. Deixe os capacitores na horizontal para que
	fique melhor apoiado na bancada, este é muito leve e pode cair com facilidade.
	
	Realizar as medidas em um tempo curto evitando que o circuito fique energizado por um longo período de tempo, pois os resistores estarão dissipando potência assim esquentando.
	
	Deve-se manter uma distância segura do circuito quando o mesmo está energizado assim evitando queimaduras e choque elétrico.
	
	
	\newpage
	\section{Análise}
	\justifying
	\subsection{Dados}
	\subsubsection{Ligação em estrela}
	\justifying
	
	\begin{table}[H]
		\centering
		\begin{tabular}{|c|c|c|c|c|c|c|}
			\hline %linha horizontal
			Sequencia & V$_L$[V] & I$_L$[A] & fp & P$_{3\phi}$[W] & Q$_{3\phi}$[Var] & S$_{3\phi}$[VA] \\
			\hline %linha horizontal
			ABC & 100 & 0.6 & 0.62 & 67.46 & 84.18 & 107.70     \\
			\hline %linha horizontal
			CBA & 100 & 0.6 & 0.62 & 67.02 & 83.59 & 107.35     \\
			\hline %linha horizontal
		\end{tabular}
		\caption{Medidas Para circuito em Estrela obtida pelo equipamento Kron}
	\end{table}
	\begin{table}[H]
		\centering
		\begin{tabular}{|c|c|c|c|}
			\hline %linha horizontal
			Sequencia & w$_1$[W] & w$_2$[W] & $W_1 + W_2$ \\
			\hline %linha horizontal
			ABC & 5 & 55 & 60     \\
			\hline %linha horizontal
			CBA & 55 & 5 & 60     \\
			\hline %linha horizontal
		\end{tabular}
		\caption{Medidas Para circuito em Estrela}
	\end{table}
	
	Para Sequencia ABC $W_2$ corresponde ao $W_1$ teórico então, temos que:
	
	\[\begin{split}
		Q_{3\phi} & = \sqrt{3}\ast(w_2 - w_1) \\
		Q_{3\phi} & = \sqrt{3}\ast(55 - 5 ) \\
		Q_{3\phi} & = 86.6
	\end{split}
	\]
	
	Para Sequencia CBA $W_1$ corresponde ao $W_1$ teórico então, temos que:
	\[\begin{split}
		Q_{3\phi} & = \sqrt{3}\ast(w_1 - w_2) \\
		Q_{3\phi} & = \sqrt{3}\ast(55 - 5 ) \\
		Q_{3\phi} & = 86.6
	\end{split}
	\]
	
	\subsubsection{Ligação em delta}
	\justifying
	\begin{table}[H]
		\centering
		\begin{tabular}{|c|c|c|c|c|c|c|c|}
			\hline %linha horizontal
			Sequencia & V$_L$[V] & I$_L$[A] & I$_f$[A] & fp & P$_{3\phi}$[W] & Q$_{3\phi}$[Var] & S$_{3\phi}$[VA] \\
			\hline %linha horizontal
			ABC & 81 & 1.8 & 1 & 0.658 & 168.44 & 193.01 & 257.95     \\
			\hline %linha horizontal
			CBA & 80 & 1.8 & 1 & 0.657 & 164 & 188.29 & 249.89     \\
			\hline %linha horizontal
		\end{tabular}
		\caption{Medidas Para circuito em Delta obtidas pelo equipamento Kron}
	\end{table}
	
	\begin{table}[H]
		\centering
		\begin{tabular}{|c|c|c|c|}
			\hline %linha horizontal
			Sequencia & w$_1$[W] & w$_2$[W] & $W_1 + W_2$ \\
			\hline %linha horizontal
			ABC & 115 & 35 & 150     \\
			\hline %linha horizontal
			CBA & 15 & 140 & 155     \\
			\hline %linha horizontal
		\end{tabular}
		\caption{Medidas Para circuito em Delta}
	\end{table}
	Para Sequencia ABC $W_2$ corresponde ao $W_1$ teórico então, temos que:
	
	\[\begin{split}
		Q_{3\phi} & = \sqrt{3}\ast(w_2 - w_1) \\
		Q_{3\phi} & = \sqrt{3}\ast(35 - 115 ) \\
		Q_{3\phi} & = -138.6 \text{ V}_{ar}
	\end{split}
	\]
	
	Para Sequencia CBA $W_1$ corresponde ao $W_1$ teórico então, temos que:
	\[\begin{split}
		Q_{3\phi} & = \sqrt{3}\ast(w_1 - w_2) \\
		Q_{3\phi} & = \sqrt{3}\ast(15 - 140 ) \\
		Q_{3\phi} & = -216.5 \text{ V}_{ar}
	\end{split}
	\]
	
	\subsection{Questões}
	\justifying
	- Para os sistemas das Figuras 1 e 2, ao ser ligado, o que aconteceu com os wattímetros
	$W_1$ e $W_2$ quando a sequência de fases foi invertida? Algum deles marcou valor
	negativo? Explique. Encontre as potências usando as leituras.\\
	R: Ao mudar a sequencia os wattímetros trocaram as leituras. Nenhum marcou e leitura negativa. O Angulo $\theta _z$ de ambas as cargas é menor que $60^\circ$ e a ligação estava correta  por isso não houve medição negativa. Potencia calculada acima.
	
	- Encontre o valor das leituras dos wattímetros usando as expressões analíticas.\\
	Para circuito estrela:\\
	
	\[\begin{split}
		&W_1 = V_L\ast I_L \ast \cos(\theta _z + 30^\circ) \\
		&W_2 = V_L\ast I_L \ast \cos(\theta _z - 30^\circ)\\
		&\cos(\theta) = 0.62 \implies \theta = 51.68 ^\circ\\
		&W_1 = 8.68\\
		&W_2 = 55.76
	\end{split}
	\]
	Para circuito delta:\\
	
	\[\begin{split}
		&W_1 = V_L\ast I_L \ast \cos(\theta _z + 30^\circ) \\
		&W_2 = V_L\ast I_L \ast \cos(\theta _z - 30^\circ)\\
		&\cos(\theta) = 0.658 \implies \theta = -48.85 ^\circ\\
		&W_1 = 136.28\\
		&W_2 = 27.85
	\end{split}
	\]
	
	- Mostre através de um diagrama fasorial que de acordo com as polaridades das bobinas
	de corrente e de potencial a leitura do wattímetro analógico é positiva para um ângulo $\left| \theta_z \right|$ menor que 60º.Mostre que a leitura será negativa se  for maior que 60º.\\
	R: Se $\left|\theta_z\right| < 60^\circ$ então $\left |\theta_z + 30^\circ\right| < 90^\circ$ e
	$\left| \theta_z - 30^\circ \right| < 90^\circ$ assim para $\theta < 90$, $\cos(\theta) > 0$ , ambas leituras serão positivas.\\
	Se $\left|\theta_z\right| > 60^\circ$ então $\left |\theta_z + 30^\circ\right| > 90^\circ$ e
	$\left| \theta_z - 30^\circ \right| < 90^\circ$ assim para $\theta > 90$, $\cos(\theta) < 0$, uma das leituras será negativa.
	
	
	
	
	
	
	- Mostre através de um diagrama fasorial que se a polaridade de uma das bobinas não
	for seguida a leitura terá um sinal oposto ao correto.\\
	R: Pela definição caso uma  das medidas fique errada, será lido a medida com angulo somado 180. conforme as figuras abaixo mostram. As figuras foram feita a partir de um código[1] em javaScript escrito pelo autor deste relatório.
	\begin{figure}[H]
		\centering % para centralizarmos a figura
		\includegraphics[width=0.7\columnwidth]{fasores.png}
		\caption{Fasores com a medição correta}
	\end{figure}
	\begin{figure}[H]
		\centering % para centralizarmos a figura
		\includegraphics[width=0.7\columnwidth]{fasores2.png}
		\caption{Caso troque a medição $V_{ca}$ por $V_{ac}$ gerando uma leitura negativa}
	\end{figure}
	
	O mesmo acontece caso a bobina de corrente for invertida.
	\begin{figure}[H]
		\centering % para centralizarmos a figura
		\includegraphics[width=0.7\columnwidth]{fasoresI1.png}
		\caption{Fasores de corrente com a medição correta}
	\end{figure}
	\begin{figure}[H]
		\centering % para centralizarmos a figura
		\includegraphics[width=0.7\columnwidth]{fasoresI2.png}
		\caption{Caso troque a medição $I_{a}$ por $-I_a$ gerando uma leitura negativa}
	\end{figure}
	
	\newpage
	\section{Simulação}
	\justifying
	\begin{figure}[H]
		\centering % para centralizarmos a figura
		\includegraphics[width=0.7\columnwidth]{simulacao/estrelaabc.png}
		\caption{Simulação em estrela ABC}
	\end{figure}
	\begin{figure}[H]
		\centering % para centralizarmos a figura
		\includegraphics[width=0.7\columnwidth]{simulacao/estrelacba.png}
		\caption{Simulação em estrela CBA}
	\end{figure}
	\begin{figure}[H]
		\centering % para centralizarmos a figura
		\includegraphics[width=0.7\columnwidth]{simulacao/simulacaodelta.png}
		\caption{Simulação em delta ABC}
	\end{figure}
	\begin{figure}[H]
		\centering % para centralizarmos a figura
		\includegraphics[width=0.7\columnwidth]{simulacao/simulacaodeltacba.png}
		\caption{Simulação em estrela CBA}
	\end{figure}
	\begin{table}[H]
		\centering
		\begin{tabular}{|c|c|c|c|c|c|}
			\hline %linha horizontal
			Sequência & V$_L$[V] & I$_L$[A] & w$_1$[W] & w$_2$[W] & $W_1 + W_2$ \\
			\hline %linha horizontal
			ABC & 80 & 1.8 & 28.1 & 140 & 168     \\
			\hline %linha horizontal
			CBA & 80 & 1.8 & 140 & 28.1 & 168     \\
			\hline %linha horizontal
		\end{tabular}
		\caption{Medidas Simuladas  Para circuito em Delta}
	\end{table}
	
	\begin{table}[H]
		\centering
		\begin{tabular}{|c|c|c|c|c|c|}
			\hline %linha horizontal
			Sequência & V$_L$[V] & I$_L$[A] & w$_1$[W] & w$_2$[W] & $W_1 + W_2$ \\
			\hline %linha horizontal
			ABC & 100 & 0.7 & 12.3 & 69.1 & 81.4     \\
			\hline %linha horizontal
			CBA & 100 & 0.7 & 69.1 & 12.3 & 81.4     \\
			\hline %linha horizontal
		\end{tabular}
		\caption{Medidas Simuladas Para circuito em Estrela}
	\end{table}
	
	
	
	
	
	
	\newpage
	\section{Conclusão}
	\justifying
	A medição de potencia trifásica usando o método dos dois wattímetros é importante para conhecimentos didáticos porém não é comumente utilizado em meios práticos. Com o desenvolvimento da tecnologia foram feitos equipamentos digitais mais precisos e mais fáceis de manusear, estes também realizam outras medições.
	\begin{figure}[H]
		\centering % para centralizarmos a figura
		\includegraphics[width=0.7\columnwidth]{pttr.jpg}
		\caption{Analisador de potência trifásica digital [2]}
	\end{figure}
	
	Os valores analíticos foram levemente diferentes dos obtidos experimentalmente, quando é feito os cálculos despreza-se os efeitos das bobinas.
	
	Foi observado que ao trocar as sequencias de fases as medições inverteram, porém para quando o circuito era capacitivo as medições inverteram e  foram diferentes, que leva a levantar a hipótese que os capacitores possuíam uma diferença considerável  fazendo o circuito distanciar de um circuito perfeitamente equilibrado.
	
	A simulação mostrou que ao invertei a sequencia de fase as medições dos wattímetros mudam, este também comprovando a teoria.
	
	O método mostrou funcionar independente da ligação (estrela ou delta) e independente do tipo da carga (capacitiva ou indutiva) bastando ser equilibrada.
	
	\newpage
	\section*{Referencias}
	\justifying
	\noindent
	ALEXANDER, C.K.; SADIKU, M.N. Fundamentos de Circuitos Elétricos. 5ª ed.
	Porto Alegre: Mc Graw-Hill, 2015\\
	
	\noindent
	[1] - LIMA, H.S.; Desenhando Fasores https://xx220xx.github.io/FASORES/index.html acesso em 28/10/2018 \\
	
	\noindent
	[2] Analisador de potência trifásico PCE-PA 8000 - https://www.pce-medidores.com.pt/fichas-dados/analisador-potencia-trifasico-pce-pa-8000.htm acesso em 28/10/2018
\end{document}



