\documentclass[a4paper, 12pt]{article}


\usepackage[portuges]{babel}
\usepackage[utf8]{inputenc}

\usepackage{indentfirst}
\usepackage{graphicx}
\usepackage{multicol,lipsum}
\usepackage{mathptmx}
\usepackage{setspace} % espaçamento entre linhas

% padrao 1.5 de espacamento entre linhas
\setstretch{1.5}
\begin{document}

%\maketitle

\begin{titlepage}
	\begin{center}
	
	%\begin{figure}[!ht]
	%\centering
	%\includegraphics[width=2cm]{c:/ufba.jpg}
	%\end{figure}

		\Huge{Universidade Federal de Uberlândia}\\
	
        \textbf{\LARGE{}}\\
		%\title{{\large{Título}}}
		\vspace{3,5cm}
	\end{center}
	
	\begin{flushleft}
		\begin{tabbing}
			Aluno: Henrique Santos de Lima - 11811ETE016\\
			Professor: Wellington Maycon Santos Bernardes\\
	\end{tabbing}
 \end{flushleft}
	\vspace{1cm}
	
	\begin{center}
		\vspace{\fill}
			 Setembro\\
		 2019
			\end{center}
\end{titlepage}
%%%%%%%%%%%%%%%%%%%%%%%%%%%%%%%%%%%%%%%%%%%%%%%%%%%%%%%%%%%

% % % % % % % % %FOLHA DE ROSTO % % % % % % % % % %

\begin{titlepage}
	\begin{center}
	
	%\begin{figure}[!ht]
	%\centering
	%\includegraphics[width=2cm]{c:/ufba.jpg}
	%\end{figure}

		\Huge{Universidade Federal de Uberlândia}\\
	.\vspace{15pt}
        
        \vspace{85pt}
        
		\textbf{\LARGE{Relatório de Experimental de Circuitos Elétricos 2}}
		\title{\large{Título}}
	%	\large{Modelo\\
     %   		Validação do modelo clássico}
			
	\end{center}
\vspace{1,5cm}
	
	\begin{flushright}

   \begin{list}{}{
      \setlength{\leftmargin}{4.5cm}
      \setlength{\rightmargin}{0cm}
      \setlength{\labelwidth}{0pt}
      \setlength{\labelsep}{\leftmargin}}

      \item 
      Tensão e corrente de curto-circuito em regulador de tensão senoidal

      \begin{list}{}{
      \setlength{\leftmargin}{0cm}
      \setlength{\rightmargin}{0cm}
      \setlength{\labelwidth}{0pt}
      \setlength{\labelsep}{\leftmargin}}

			\item Aluno:  Henrique Santos de Lima - 11811ETE016\
            \item Professor : Wellington Maycon Santos Bernardes\
      		

      \end{list}
   \end{list}
\end{flushright}
\vspace{1cm}
\begin{center}
		\vspace{\fill}
			 Setembro\\
		 2019
			\end{center}
\end{titlepage}

\tableofcontents

\thispagestyle{empty}

\newpage
\pagenumbering{times new romam}
% % % % % % % % % % % % % % % % % % % % % % % % % % %
\section{Objetivos}
\mbox{}
Montar um circuito sob curto circuito, energizá-lo com tensão alternada senoidal e realizar medições usando equipamentos analógicos e digitais. Mostrar a importância de manusear com cautela o regulador de tensão para identificação de curto-circuito nos primeiros instantes assim evitando danificar aparelhos.
\section{Introdução}
\mbox{}
Diz que o circuito está em curto quando o mesmo possui um percusso fechado sem resistência, assim a corrente fica muito elevada e a tensão permanece em zero, isso teoricamente, pois em condições reais o fio possui resistência , esta muito baixa porém não nula. Esta situação pode causar grandes danos ao equipamentos pois a potência dissipada pelo fio  é muito elevada podendo ate causar incêndios.


\newpage
\section{Preparação}
\mbox{}
\subsection{Materiais e ferramentas}
\mbox{}

\newpage
\section{Análise sobre segurança}

\newpage
\section{Análise}
\newpage
\section{Simulação}
\newpage
\section{Conclusões}
\newpage
\addcontentsline{toc}{section}{Bibliografia}
\section*{Referencias}
\newpage
\end{document}



